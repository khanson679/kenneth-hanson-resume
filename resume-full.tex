% arara: pdflatex
%
% Kenneth Hanson's Resume
% Created: 6/4/2012
% Based on Andrew McNabb's template (http://www.mcnabbs.org/andrew/linux/latexres/)

\newcommand{\editdate}{04/06/2020}

\documentclass[10pt,oneside]{article}

\usepackage[T1]{fontenc}
\usepackage{lmodern}


%%%%%%%%%%%%%%%%%%%%%%%%%%%%%%%%%%%%%%%%%%%%%%%%%%%%%%%%%
% Page formatting

\usepackage{geometry}
\geometry{
	letterpaper,
	includeheadfoot, % avoids conflict with fancyheader
	hmargin    = 0.5in,
	vmargin    = 0.5in,
	headheight = 0in,
	headsep    = 0in,
	footskip   = 0.3in
}

\usepackage{fancyhdr}
\usepackage{lastpage}
\pagestyle{fancy}
\renewcommand{\headrulewidth}{0pt} % remove header rule
\fancyhf{} % clear header and footer
\fancyfoot[L]{Last Updated \editdate}
\fancyfoot[C]{\thepage/\pageref*{LastPage}}

%%%%%%%%%%%%%%%%%%%%%%%%%%%%%%%%%%%%%%%%%%%%%%%%%%%%%%%%%
% Other global formatting

\setlength{\parindent}{0pt}

\usepackage[sf]{titlesec}

\usepackage{enumitem}
\setlist[itemize]{parsep=0pt,leftmargin=15pt,label=\raisebox{0.4ex}{\tiny\textbullet}}

\usepackage{hyperref}
\hypersetup{colorlinks=true,urlcolor=blue}
\urlstyle{same} % don't use typewriter font


%%%%%%%%%%%%%%%%%%%%%%%%%%%%%%%%%%%%%%%%%%%%%%%%%%%%%%%%%
% New commands and environments

% Name and contact information
\newcommand{\name}{Kenneth Hanson}
\newcommand{\email}{khanson679@gmail.com}
\newcommand{\website}{khanson679.github.io}


% Layout for top heading
\newcommand{\bigname}{%
	{\centering\scshape\Huge \name \par}
}
\newcommand{\contactinfo}{%
	{\raggedleft\small \href{http://\website}{\website} | \href{mailto:\email}{\email} \par}
}
\newcommand{\topheading}{%
	\bigname
	\rule{\textwidth}{1pt} \par
	\contactinfo
}


% list formatting for resume items
\newlist{reslist}{itemize}{2}
\setlist[reslist]{parsep=0pt}
\setlist[reslist,1]{leftmargin=0pt,itemsep=6pt,label=}
%\setlist[reslist,2]{leftmargin=15pt,label=\raisebox{0.4ex}{\tiny\textbullet}}


% section commands
\newcommand{\ressection}[1]{
	\vspace{-12pt}
	\section*{#1}
}
\newcommand{\ressubsection}[1]{
	\subsection*{#1}
	\vspace{-4pt}
}


% \resitem -- item with associated date
%   date is right aligned, item has simple text wrapping
%   Arg 1: Item
%   Arg 2: Date
\newcommand{\resitem}[2]{
	\item \parbox[t]{0.8\textwidth}{#1} \hfill #2
}


% \resbigitem -- item with institution, date, role
%  Arg 1: Role
%  Arg 2: Date
%  Arg 3: Institution/employer/etc.
%  Arg 4: Date range
\newcommand{\resbigitem}[4]{
	\item \parbox[t]{0.8\textwidth}{\textbf{#1}} \hfill #2 \\
		\parbox[t]{0.8\textwidth}{\textit{#3}} \hfill \textit{#4}
}


% list formatting for publications and presentations
\newlist{publist}{description}{2}
\setlist[publist]{parsep=0pt,font=\normalfont}


%%%%%%%%%%%%%%%%%%%%%%%%%%%%%%%%%%%%%%%%%%%%%%%%%%%%%%%%%
\begin{document}

\topheading

\ressection{Education}

\begin{reslist}
	\resbigitem{B.A., Linguistics}
		{01/2011--05/2014}
		{Michigan State University, East Lansing, MI}
		{}
	\begin{itemize}
		\item Additional Major in Japanese. Minor in Computer Science. Cumulative GPA: 4.0.
		\item Coursework in graduate syntactic theory, Japanese linguistics, Classical Japanese, compilers, and software design.
%		\item Independent study project using automated morphological analysis and Python and R scripts to analyze the syntax of Classical Japanese poetry. Abstract available on website.
		\item Transferred 15 cr. from Eastern Michigan University (Ypsilanti, MI), and 38 AP credits.
	\end{itemize}
	
	\resbigitem{2013 LSA Linguistic Institute}
		{06/2013--07/2013}
		{University of Michigan, Ann Arbor, MI}
		{}
	\begin{itemize}
		\item Coursework in comp.\ linguistics, comp.\ psycholinguistics, and language typology.
		\item Attended workshops on diachronic syntax and syntactic variation.
	\end{itemize}
\end{reslist}


\ressection{Positions}

\begin{reslist}
	\resbigitem{Assistant English Teacher}
		{08/2014-08/2018}
		{Japan Exchange and Teaching (JET) Programme}
		{}
	\begin{itemize}
		\item Taught students (25-40 per class) in grades 1--9 at two junior high schools and four elementary schools. Lessons primarily team-taught with each class's primary instructor.
		\item Developed custom curriculum for grades 1--4. Designed activities to accompany curriculum for grades 5--9.
		\item Designed flashcards, worksheets, presentations, printed game materials, and game software using LibreOffice, LaTeX, ConTeXt, Python, and Qt.
	\end{itemize}

	\resbigitem{Undergraduate Research Assistant}
		{09/2012--05/2014}
		{For Cristina Schmitt, PhD and Alan Munn, PhD, MSU Dept.\ of Linguistics and Languages}
		{}
	\begin{itemize}
		\item Conducted a diachronic corpus study of bare noun phrases in English using CorpusSearch, Python, and R.
		\item Wrote a program to convert CorpusSearch coding output to spreadsheet form for data analysis.
		\item Scheduled and ran participants for an artificial language learning experiment.
		\item Partially funded by the MSU College of Arts and Letters Undergraduate Research Initiative (CAL-URI).
	\end{itemize}
	
	\resbigitem{Undergraduate Research Assistant}
		{05/2012--06/2013}
		{For Joyce Chai, PhD, MSU Language and Interaction Research Group (LAIR)}
		{}
	\begin{itemize}
		\item Worked with lab members to design a natural language dialog system for a humanoid robot. Developed basic software architecture, AI modules, and debug interface. Coordinated integration of other modules. All programming in Python.
		\item Installed and configured Ubuntu Linux and software for development and robot operation. Managed group documentation.
		\item Funded by the MSU Engineering Summer Undergraduate Research Experience Program (EnSURE) for Summer 2012.
	\end{itemize}
	
	\resbigitem{Undergraduate Teaching Assistant}
		{01/2012-05/2013}
		{For CSE 232 (Intro to Programming II), MSU Dept.\ of Computer Science and Engineering}
		{}
	\begin{itemize}
		\item Taught short lessons on C++ and data structures, and guided students (avg.\ 17 per semester) through assignments during weekly lab sessions.
		\item Tutored students for and graded programming projects.
		\item Produced web materials and handouts including C++ examples and guides on Linux and programming.
	\end{itemize}
\end{reslist}


\ressection{Publications}

\begin{publist}
	\item[2014] {Chai, J. Y., L. She, R. Fang, S. Ottarson, C. Littley, C. Liu, and K. Hanson. Collaborative Effort towards Common Ground in Situated Human Robot Dialogue. 2014 ACM/IEEE International Conference on Human-Robot Interaction (HRI). Bielefeld, Germany.}
\end{publist}


\ressection{Presentations}

\ressubsection{Refereed Conference Posters}
\begin{publist}
	\item[2014] {Hanson, Kenneth, Cristina Schmitt, and Alan Munn. The loss of bare singular arguments and predicates in the history of English. 16th Diachronic Generative Syntax (DiGS) Conference. Budapest, Hungary.}
\end{publist}

\ressubsection{Refereed Undergraduate Conference Talks}
\begin{publist}
	\item[2014] {Hanson, Kenneth, Cristina Schmitt, and Alan Munn. The loss of bare singular noun phrases in the history of English. Great Lakes Expo for Experimental and Formal Undergraduate Linguistics (GLEEFUL). Michigan State University, East Lansing, MI.}
\end{publist}

\ressubsection{Other Undergraduate Conference Presentations}
\begin{publist}
	\item[2014] {Hanson, Kenneth. Quantitative methods for the analysis of Classical Japanese poetry. Michigan State Undergraduate Linguistics Conference (MSULC). Michigan State University, East Lansing, MI.}
	
	\item[2014] {Hanson, Kenneth. Methods for tracking lexical classes in parsed historical corpora. Poster presented at the Michigan State Undergraduate Linguistics Conference (MSULC). Michigan State University, East Lansing, MI.}
	
	\item[2013] {Hanson, Kenneth. CorpusExtract: a tool for analyzing syntactically annotated corpora. Poster presented at the Michigan State Undergraduate Linguistics Conference (MSULC). Michigan State University, East Lansing, MI.}
	
	\item[2012] {Hanson, Kenneth. Playing a naming game with Darwin: towards human-robot dialog. Summer Undergraduate Research Forum (SURF). Michigan State University, East Lansing, MI.}
\end{publist}


\ressection{Other Activities}

\begin{reslist}
	\resbigitem{Author and Webmaster}
		{06/2011--Present}
		{JapaneseProfessor.com}
		{}
	\begin{itemize}
		\item Created beginning Japanese curriculum including 33 lessons on grammar, vocabulary, writing, and culture.
	\end{itemize}
	
	\resbigitem{Vice President, Secretary, and Activities Director}
		{02/2012--05/2014}
		{q Undergraduate Association for Linguistics at Michigan State (qUALMS)}
		{}
	\begin{itemize}
		\item Other positions held: advertising, tech support. Organized and ran activities at biweekly meetings. Produced workshop handouts and fliers for advertising. Rebuilt and maintained club website.
		\item Produced programs for, helped organize and run undergraduate linguistics conferences MSULC and GLEEFUL. 
	\end{itemize}
\end{reslist}


\ressection{Awards and Honors}
	
\ressubsection{Funded Awards}
\begin{reslist}
	\resbigitem{Board of Trustees Scholarship Award}
		{05/2014}
		{Michigan State University}
		{}
	\resbigitem{Outstanding Senior Achievement Award}
		{05/2014}
		{MSU College of Arts and Letters }
		{}
	\resbigitem{Engineering Summer Undergraduate Research Experience Program (EnSURE)}
		{05-07/2012}
		{MSU College of Engineering}
		{}
	\resbigitem{Mitsui US/Community Scholarship for YFU Summer Japan Study Abroad Program}
		{06--07/2009}
		{Mitsui USA Foundation}
		{}
\end{reslist}

\ressubsection{Other Awards}
\begin{reslist}
	\resbigitem{Presentation Award}
		{04/2014}
		{MSULC, Michigan State University, East Lansing, MI}
		{}
\end{reslist}

%\ressubsection{Honor Societies}
%\begin{itemize}
%	\resitem{Phi Beta Kappa}{2014--Present}
%	\resitem{AAJT Japanese National Honor Society College Chapter}{2014--Present}
%\end{itemize}


\ressection{Skills and Additional Information}

\begin{itemize}
	\item \textbf{Languages} -- English (native), Japanese (advanced, passed JLPT N2 in Dec. 2013).
	
	\item \textbf{Programming} -- Experienced: Python, C++. Familiar: Java, R, Qt, wxWidgets, ANTLR3, Make, Git, SVN. Basic knowledge: Bash, x86, SPARC.
	
	\item \textbf{Misc.\ Computing} -- Experienced: Ubuntu Linux, \LaTeX{}, CorpusSearch. Familiar: Wordpress, HTML/CSS, ConTeXt.
	
	\item \textbf{Music} --  Piano (advanced), marimba (advanced), and some composition.
	
	\item \textbf{Volunteer} -- Michigan Japanese Quiz Bowl (03/2012--03/2014)
\end{itemize}


\end{document}
